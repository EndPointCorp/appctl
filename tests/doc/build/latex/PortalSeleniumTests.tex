% Generated by Sphinx.
\def\sphinxdocclass{report}
\documentclass[letterpaper,10pt,english]{sphinxmanual}
\usepackage[utf8]{inputenc}
\DeclareUnicodeCharacter{00A0}{\nobreakspace}
\usepackage{cmap}
\usepackage[T1]{fontenc}
\usepackage{babel}
\usepackage{times}
\usepackage[Bjarne]{fncychap}
\usepackage{longtable}
\usepackage{sphinx}
\usepackage{multirow}



\title{Portal Selenium Tests Documentation}
\date{November 28, 2014}
\release{0.1}
\author{End Point}
\newcommand{\sphinxlogo}{}
\renewcommand{\releasename}{Release}
\makeindex

\makeatletter
\def\PYG@reset{\let\PYG@it=\relax \let\PYG@bf=\relax%
    \let\PYG@ul=\relax \let\PYG@tc=\relax%
    \let\PYG@bc=\relax \let\PYG@ff=\relax}
\def\PYG@tok#1{\csname PYG@tok@#1\endcsname}
\def\PYG@toks#1+{\ifx\relax#1\empty\else%
    \PYG@tok{#1}\expandafter\PYG@toks\fi}
\def\PYG@do#1{\PYG@bc{\PYG@tc{\PYG@ul{%
    \PYG@it{\PYG@bf{\PYG@ff{#1}}}}}}}
\def\PYG#1#2{\PYG@reset\PYG@toks#1+\relax+\PYG@do{#2}}

\expandafter\def\csname PYG@tok@gd\endcsname{\def\PYG@tc##1{\textcolor[rgb]{0.63,0.00,0.00}{##1}}}
\expandafter\def\csname PYG@tok@gu\endcsname{\let\PYG@bf=\textbf\def\PYG@tc##1{\textcolor[rgb]{0.50,0.00,0.50}{##1}}}
\expandafter\def\csname PYG@tok@gt\endcsname{\def\PYG@tc##1{\textcolor[rgb]{0.00,0.27,0.87}{##1}}}
\expandafter\def\csname PYG@tok@gs\endcsname{\let\PYG@bf=\textbf}
\expandafter\def\csname PYG@tok@gr\endcsname{\def\PYG@tc##1{\textcolor[rgb]{1.00,0.00,0.00}{##1}}}
\expandafter\def\csname PYG@tok@cm\endcsname{\let\PYG@it=\textit\def\PYG@tc##1{\textcolor[rgb]{0.25,0.50,0.56}{##1}}}
\expandafter\def\csname PYG@tok@vg\endcsname{\def\PYG@tc##1{\textcolor[rgb]{0.73,0.38,0.84}{##1}}}
\expandafter\def\csname PYG@tok@m\endcsname{\def\PYG@tc##1{\textcolor[rgb]{0.13,0.50,0.31}{##1}}}
\expandafter\def\csname PYG@tok@mh\endcsname{\def\PYG@tc##1{\textcolor[rgb]{0.13,0.50,0.31}{##1}}}
\expandafter\def\csname PYG@tok@cs\endcsname{\def\PYG@tc##1{\textcolor[rgb]{0.25,0.50,0.56}{##1}}\def\PYG@bc##1{\setlength{\fboxsep}{0pt}\colorbox[rgb]{1.00,0.94,0.94}{\strut ##1}}}
\expandafter\def\csname PYG@tok@ge\endcsname{\let\PYG@it=\textit}
\expandafter\def\csname PYG@tok@vc\endcsname{\def\PYG@tc##1{\textcolor[rgb]{0.73,0.38,0.84}{##1}}}
\expandafter\def\csname PYG@tok@il\endcsname{\def\PYG@tc##1{\textcolor[rgb]{0.13,0.50,0.31}{##1}}}
\expandafter\def\csname PYG@tok@go\endcsname{\def\PYG@tc##1{\textcolor[rgb]{0.20,0.20,0.20}{##1}}}
\expandafter\def\csname PYG@tok@cp\endcsname{\def\PYG@tc##1{\textcolor[rgb]{0.00,0.44,0.13}{##1}}}
\expandafter\def\csname PYG@tok@gi\endcsname{\def\PYG@tc##1{\textcolor[rgb]{0.00,0.63,0.00}{##1}}}
\expandafter\def\csname PYG@tok@gh\endcsname{\let\PYG@bf=\textbf\def\PYG@tc##1{\textcolor[rgb]{0.00,0.00,0.50}{##1}}}
\expandafter\def\csname PYG@tok@ni\endcsname{\let\PYG@bf=\textbf\def\PYG@tc##1{\textcolor[rgb]{0.84,0.33,0.22}{##1}}}
\expandafter\def\csname PYG@tok@nl\endcsname{\let\PYG@bf=\textbf\def\PYG@tc##1{\textcolor[rgb]{0.00,0.13,0.44}{##1}}}
\expandafter\def\csname PYG@tok@nn\endcsname{\let\PYG@bf=\textbf\def\PYG@tc##1{\textcolor[rgb]{0.05,0.52,0.71}{##1}}}
\expandafter\def\csname PYG@tok@no\endcsname{\def\PYG@tc##1{\textcolor[rgb]{0.38,0.68,0.84}{##1}}}
\expandafter\def\csname PYG@tok@na\endcsname{\def\PYG@tc##1{\textcolor[rgb]{0.25,0.44,0.63}{##1}}}
\expandafter\def\csname PYG@tok@nb\endcsname{\def\PYG@tc##1{\textcolor[rgb]{0.00,0.44,0.13}{##1}}}
\expandafter\def\csname PYG@tok@nc\endcsname{\let\PYG@bf=\textbf\def\PYG@tc##1{\textcolor[rgb]{0.05,0.52,0.71}{##1}}}
\expandafter\def\csname PYG@tok@nd\endcsname{\let\PYG@bf=\textbf\def\PYG@tc##1{\textcolor[rgb]{0.33,0.33,0.33}{##1}}}
\expandafter\def\csname PYG@tok@ne\endcsname{\def\PYG@tc##1{\textcolor[rgb]{0.00,0.44,0.13}{##1}}}
\expandafter\def\csname PYG@tok@nf\endcsname{\def\PYG@tc##1{\textcolor[rgb]{0.02,0.16,0.49}{##1}}}
\expandafter\def\csname PYG@tok@si\endcsname{\let\PYG@it=\textit\def\PYG@tc##1{\textcolor[rgb]{0.44,0.63,0.82}{##1}}}
\expandafter\def\csname PYG@tok@s2\endcsname{\def\PYG@tc##1{\textcolor[rgb]{0.25,0.44,0.63}{##1}}}
\expandafter\def\csname PYG@tok@vi\endcsname{\def\PYG@tc##1{\textcolor[rgb]{0.73,0.38,0.84}{##1}}}
\expandafter\def\csname PYG@tok@nt\endcsname{\let\PYG@bf=\textbf\def\PYG@tc##1{\textcolor[rgb]{0.02,0.16,0.45}{##1}}}
\expandafter\def\csname PYG@tok@nv\endcsname{\def\PYG@tc##1{\textcolor[rgb]{0.73,0.38,0.84}{##1}}}
\expandafter\def\csname PYG@tok@s1\endcsname{\def\PYG@tc##1{\textcolor[rgb]{0.25,0.44,0.63}{##1}}}
\expandafter\def\csname PYG@tok@gp\endcsname{\let\PYG@bf=\textbf\def\PYG@tc##1{\textcolor[rgb]{0.78,0.36,0.04}{##1}}}
\expandafter\def\csname PYG@tok@sh\endcsname{\def\PYG@tc##1{\textcolor[rgb]{0.25,0.44,0.63}{##1}}}
\expandafter\def\csname PYG@tok@ow\endcsname{\let\PYG@bf=\textbf\def\PYG@tc##1{\textcolor[rgb]{0.00,0.44,0.13}{##1}}}
\expandafter\def\csname PYG@tok@sx\endcsname{\def\PYG@tc##1{\textcolor[rgb]{0.78,0.36,0.04}{##1}}}
\expandafter\def\csname PYG@tok@bp\endcsname{\def\PYG@tc##1{\textcolor[rgb]{0.00,0.44,0.13}{##1}}}
\expandafter\def\csname PYG@tok@c1\endcsname{\let\PYG@it=\textit\def\PYG@tc##1{\textcolor[rgb]{0.25,0.50,0.56}{##1}}}
\expandafter\def\csname PYG@tok@kc\endcsname{\let\PYG@bf=\textbf\def\PYG@tc##1{\textcolor[rgb]{0.00,0.44,0.13}{##1}}}
\expandafter\def\csname PYG@tok@c\endcsname{\let\PYG@it=\textit\def\PYG@tc##1{\textcolor[rgb]{0.25,0.50,0.56}{##1}}}
\expandafter\def\csname PYG@tok@mf\endcsname{\def\PYG@tc##1{\textcolor[rgb]{0.13,0.50,0.31}{##1}}}
\expandafter\def\csname PYG@tok@err\endcsname{\def\PYG@bc##1{\setlength{\fboxsep}{0pt}\fcolorbox[rgb]{1.00,0.00,0.00}{1,1,1}{\strut ##1}}}
\expandafter\def\csname PYG@tok@kd\endcsname{\let\PYG@bf=\textbf\def\PYG@tc##1{\textcolor[rgb]{0.00,0.44,0.13}{##1}}}
\expandafter\def\csname PYG@tok@ss\endcsname{\def\PYG@tc##1{\textcolor[rgb]{0.32,0.47,0.09}{##1}}}
\expandafter\def\csname PYG@tok@sr\endcsname{\def\PYG@tc##1{\textcolor[rgb]{0.14,0.33,0.53}{##1}}}
\expandafter\def\csname PYG@tok@mo\endcsname{\def\PYG@tc##1{\textcolor[rgb]{0.13,0.50,0.31}{##1}}}
\expandafter\def\csname PYG@tok@mi\endcsname{\def\PYG@tc##1{\textcolor[rgb]{0.13,0.50,0.31}{##1}}}
\expandafter\def\csname PYG@tok@kn\endcsname{\let\PYG@bf=\textbf\def\PYG@tc##1{\textcolor[rgb]{0.00,0.44,0.13}{##1}}}
\expandafter\def\csname PYG@tok@o\endcsname{\def\PYG@tc##1{\textcolor[rgb]{0.40,0.40,0.40}{##1}}}
\expandafter\def\csname PYG@tok@kr\endcsname{\let\PYG@bf=\textbf\def\PYG@tc##1{\textcolor[rgb]{0.00,0.44,0.13}{##1}}}
\expandafter\def\csname PYG@tok@s\endcsname{\def\PYG@tc##1{\textcolor[rgb]{0.25,0.44,0.63}{##1}}}
\expandafter\def\csname PYG@tok@kp\endcsname{\def\PYG@tc##1{\textcolor[rgb]{0.00,0.44,0.13}{##1}}}
\expandafter\def\csname PYG@tok@w\endcsname{\def\PYG@tc##1{\textcolor[rgb]{0.73,0.73,0.73}{##1}}}
\expandafter\def\csname PYG@tok@kt\endcsname{\def\PYG@tc##1{\textcolor[rgb]{0.56,0.13,0.00}{##1}}}
\expandafter\def\csname PYG@tok@sc\endcsname{\def\PYG@tc##1{\textcolor[rgb]{0.25,0.44,0.63}{##1}}}
\expandafter\def\csname PYG@tok@sb\endcsname{\def\PYG@tc##1{\textcolor[rgb]{0.25,0.44,0.63}{##1}}}
\expandafter\def\csname PYG@tok@k\endcsname{\let\PYG@bf=\textbf\def\PYG@tc##1{\textcolor[rgb]{0.00,0.44,0.13}{##1}}}
\expandafter\def\csname PYG@tok@se\endcsname{\let\PYG@bf=\textbf\def\PYG@tc##1{\textcolor[rgb]{0.25,0.44,0.63}{##1}}}
\expandafter\def\csname PYG@tok@sd\endcsname{\let\PYG@it=\textit\def\PYG@tc##1{\textcolor[rgb]{0.25,0.44,0.63}{##1}}}

\def\PYGZbs{\char`\\}
\def\PYGZus{\char`\_}
\def\PYGZob{\char`\{}
\def\PYGZcb{\char`\}}
\def\PYGZca{\char`\^}
\def\PYGZam{\char`\&}
\def\PYGZlt{\char`\<}
\def\PYGZgt{\char`\>}
\def\PYGZsh{\char`\#}
\def\PYGZpc{\char`\%}
\def\PYGZdl{\char`\$}
\def\PYGZhy{\char`\-}
\def\PYGZsq{\char`\'}
\def\PYGZdq{\char`\"}
\def\PYGZti{\char`\~}
% for compatibility with earlier versions
\def\PYGZat{@}
\def\PYGZlb{[}
\def\PYGZrb{]}
\makeatother

\renewcommand\PYGZsq{\textquotesingle}

\begin{document}

\maketitle
\tableofcontents
\phantomsection\label{index::doc}


Description of the test modules, classes and particular test cases
in the \textbf{tests/extensions/functional/tests} portal project package.

The test cases have \textbf{test\_} prefix in their names.


\chapter{tests/extensions/functional/tests package}
\label{modules:tests-extensions-functional-tests-package}\label{modules::doc}\label{modules:portal-selenium-tests-documentation}

\section{test\_display\_kiosk module}
\label{test_display_kiosk:module-test_display_kiosk}\label{test_display_kiosk::doc}\label{test_display_kiosk:test-display-kiosk-module}\index{test\_display\_kiosk (module)}
The display extension tests.

The kiosk extension tests.

Cross-verification of DOM elements between the display and kiosk extensions.
\index{TestBaseDisplay (class in test\_display\_kiosk)}

\begin{fulllineitems}
\phantomsection\label{test_display_kiosk:test_display_kiosk.TestBaseDisplay}\pysigline{\strong{class }\code{test\_display\_kiosk.}\bfcode{TestBaseDisplay}}
Tests related to the display extension.
\index{test\_widgets\_not\_displayed() (test\_display\_kiosk.TestBaseDisplay method)}

\begin{fulllineitems}
\phantomsection\label{test_display_kiosk:test_display_kiosk.TestBaseDisplay.test_widgets_not_displayed}\pysiglinewithargsret{\bfcode{test\_widgets\_not\_displayed}}{\emph{*args}, \emph{**kwargs}}{}~\begin{description}
\item[{Test that the following graphical widgets (elements) are not displayed:}] \leavevmode\begin{itemize}
\item {} 
zoom in/out buttons,

\item {} 
compass,

\item {} 
google More Fun menu,

\item {} 
search field,

\item {} 
search button,

\item {} 
runway panel (Points of Interest, Famous Places).

\end{itemize}

\end{description}

After the browser loads the initial URL, it takes some time
until also the extensions are fully loaded and until the DOM is
modified accordingly.

\end{fulllineitems}


\end{fulllineitems}

\index{TestBaseKioskExtension (class in test\_display\_kiosk)}

\begin{fulllineitems}
\phantomsection\label{test_display_kiosk:test_display_kiosk.TestBaseKioskExtension}\pysigline{\strong{class }\code{test\_display\_kiosk.}\bfcode{TestBaseKioskExtension}}
Tests related to the kiosk extension.
\index{test\_widgets\_displayed() (test\_display\_kiosk.TestBaseKioskExtension method)}

\begin{fulllineitems}
\phantomsection\label{test_display_kiosk:test_display_kiosk.TestBaseKioskExtension.test_widgets_displayed}\pysiglinewithargsret{\bfcode{test\_widgets\_displayed}}{\emph{*args}, \emph{**kwargs}}{}~\begin{description}
\item[{Test that the following graphical widgets (elements) are displayed:}] \leavevmode\begin{itemize}
\item {} 
zoom in/out buttons,

\item {} 
compass,

\item {} 
google More Fun menu,

\item {} 
search field,

\item {} 
search button,

\item {} 
runway panel (Points of Interest, Famous Places).

\end{itemize}

\end{description}

If elements from the elements list are renamed, the test
\textbf{test\_widgets\_not\_displayed()} would just test that they are not present,
however they might be, but under a different name.
Test that exactly the same are present when the display
extension is not loaded, i.e. with kiosk extension loaded.

\end{fulllineitems}


\end{fulllineitems}



\section{test\_general module}
\label{test_general:test-general-module}\label{test_general::doc}\label{test_general:module-test_general}\index{test\_general (module)}
General Portal selenium tests.
\index{TestMiscellaneous (class in test\_general)}

\begin{fulllineitems}
\phantomsection\label{test_general:test_general.TestMiscellaneous}\pysigline{\strong{class }\code{test\_general.}\bfcode{TestMiscellaneous}}
Other test cases not fitting any other current category.
\index{test\_eu\_cookies\_info\_bar\_is\_hidden() (test\_general.TestMiscellaneous method)}

\begin{fulllineitems}
\phantomsection\label{test_general:test_general.TestMiscellaneous.test_eu_cookies_info_bar_is_hidden}\pysiglinewithargsret{\bfcode{test\_eu\_cookies\_info\_bar\_is\_hidden}}{\emph{*args}, \emph{**kwargs}}{}
Test that the EU cookies info bar is invisible.

And the map canvas has set top=0px,
because the cookie container is 30 px higher,
and the canvas is moved 30 px down.

\end{fulllineitems}


\end{fulllineitems}

\index{TestSearch (class in test\_general)}

\begin{fulllineitems}
\phantomsection\label{test_general:test_general.TestSearch}\pysigline{\strong{class }\code{test\_general.}\bfcode{TestSearch}}
Test class related to the search box, search button.
Various interaction scenarios.
\index{test\_no\_searchbox\_on\_other\_planets() (test\_general.TestSearch method)}

\begin{fulllineitems}
\phantomsection\label{test_general:test_general.TestSearch.test_no_searchbox_on_other_planets}\pysiglinewithargsret{\bfcode{test\_no\_searchbox\_on\_other\_planets}}{\emph{*args}, \emph{**kwargs}}{}
The test loads the initial MAPS\_URL and zooms out. After this zoom
out, the universe objects appear (Earth, Moon, Mars).

The search box should not be visible when Moon, Mars are clicked.

First Moon is clicked, disappearance of search box is verified.
Second, the Earth is clicked, verify search box appeared.
Last, Mars is clicked, disappearance of search box is verified.

\end{fulllineitems}

\index{test\_search\_clicking\_search\_button() (test\_general.TestSearch method)}

\begin{fulllineitems}
\phantomsection\label{test_general:test_general.TestSearch.test_search_clicking_search_button}\pysiglinewithargsret{\bfcode{test\_search\_clicking\_search\_button}}{\emph{*args}, \emph{**kwargs}}{}
Test that camera coordinates were changed from the initial
position to something different after performing the search.

Interacts with search box and clicks on the search button.

\end{fulllineitems}

\index{test\_search\_hitting\_return\_on\_search\_box() (test\_general.TestSearch method)}

\begin{fulllineitems}
\phantomsection\label{test_general:test_general.TestSearch.test_search_hitting_return_on_search_box}\pysiglinewithargsret{\bfcode{test\_search\_hitting\_return\_on\_search\_box}}{\emph{*args}, \emph{**kwargs}}{}
Test that camera coordinates were changed from the initial
position to something different after performing the search.

Interacts with search box and hits the return key on on the
search box.

\end{fulllineitems}

\index{test\_search\_hitting\_return\_on\_search\_button() (test\_general.TestSearch method)}

\begin{fulllineitems}
\phantomsection\label{test_general:test_general.TestSearch.test_search_hitting_return_on_search_button}\pysiglinewithargsret{\bfcode{test\_search\_hitting\_return\_on\_search\_button}}{\emph{*args}, \emph{**kwargs}}{}
Test that camera coordinates were changed from the initial
position to something different after performing the search.

Interacts with search box and sends return key event on the
search button.

\end{fulllineitems}


\end{fulllineitems}



\section{test\_google\_menu module}
\label{test_google_menu:test-google-menu-module}\label{test_google_menu::doc}\label{test_google_menu:module-test_google_menu}\index{test\_google\_menu (module)}
Google Menu related tests.
\index{TestGoogleMenu (class in test\_google\_menu)}

\begin{fulllineitems}
\phantomsection\label{test_google_menu:test_google_menu.TestGoogleMenu}\pysigline{\strong{class }\code{test\_google\_menu.}\bfcode{TestGoogleMenu}}
Google Menu tests.
\index{test\_clicking\_doodle\_item() (test\_google\_menu.TestGoogleMenu method)}

\begin{fulllineitems}
\phantomsection\label{test_google_menu:test_google_menu.TestGoogleMenu.test_clicking_doodle_item}\pysiglinewithargsret{\bfcode{test\_clicking\_doodle\_item}}{\emph{*args}, \emph{**kwargs}}{}
Test that clicking on the doodle item changes the URL to the
doodles page.

\end{fulllineitems}

\index{test\_google\_items\_are\_visible\_on\_click() (test\_google\_menu.TestGoogleMenu method)}

\begin{fulllineitems}
\phantomsection\label{test_google_menu:test_google_menu.TestGoogleMenu.test_google_items_are_visible_on_click}\pysiglinewithargsret{\bfcode{test\_google\_items\_are\_visible\_on\_click}}{\emph{*args}, \emph{**kwargs}}{}
Test that Google Menu (More fun) items are visible after clicking it.

\end{fulllineitems}

\index{test\_google\_menu\_is\_visible() (test\_google\_menu.TestGoogleMenu method)}

\begin{fulllineitems}
\phantomsection\label{test_google_menu:test_google_menu.TestGoogleMenu.test_google_menu_is_visible}\pysiglinewithargsret{\bfcode{test\_google\_menu\_is\_visible}}{\emph{*args}, \emph{**kwargs}}{}
Test that Google Menu (More fun) is displayed along with some items.

\end{fulllineitems}


\end{fulllineitems}



\section{test\_platform module}
\label{test_platform:module-test_platform}\label{test_platform::doc}\label{test_platform:test-platform-module}\index{test\_platform (module)}
Browser platform tests (webgl, drivers etc).
\index{TestPlatform (class in test\_platform)}

\begin{fulllineitems}
\phantomsection\label{test_platform:test_platform.TestPlatform}\pysigline{\strong{class }\code{test\_platform.}\bfcode{TestPlatform}}~\begin{description}
\item[{Following tests will check whether:}] \leavevmode\begin{itemize}
\item {} 
version of the browser matches the version in config.json

\end{itemize}

\item[{Check:}] \leavevmode\begin{itemize}
\item {} 
gpu\_data.clientInfo.version =\textgreater{} Chrome/37.0.2062.120

\item {} 
gpu\_data.gpuInfo.basic\_info{[}n{]} =\textgreater{} description : `Direct Rendering'

\end{itemize}

\item[{Log:}] \leavevmode\begin{itemize}
\item {} 
gpu\_data.gpuInfo.basic\_info{[}n{]} =\textgreater{} description : GL\_VERSION

\end{itemize}

\end{description}
\index{test\_get\_chrome\_gpu\_data() (test\_platform.TestPlatform method)}

\begin{fulllineitems}
\phantomsection\label{test_platform:test_platform.TestPlatform.test_get_chrome_gpu_data}\pysiglinewithargsret{\bfcode{test\_get\_chrome\_gpu\_data}}{}{}
Check if the chrome supports GPU.

\end{fulllineitems}


\end{fulllineitems}



\section{test\_ros module}
\label{test_ros:module-test_ros}\label{test_ros::doc}\label{test_ros:test-ros-module}\index{test\_ros (module)}
Tests involving ROS communication between browsers.
\index{TestBaseSingleBrowserROS (class in test\_ros)}

\begin{fulllineitems}
\phantomsection\label{test_ros:test_ros.TestBaseSingleBrowserROS}\pysigline{\strong{class }\code{test\_ros.}\bfcode{TestBaseSingleBrowserROS}}
Tests involving single browser.

Listening and asserting ROS traffic based on the actions performed in
the (kiosk extension) browser. Browser runs in one process, another
process is ROS topic listener.
\index{test\_ros\_position\_after\_search() (test\_ros.TestBaseSingleBrowserROS method)}

\begin{fulllineitems}
\phantomsection\label{test_ros:test_ros.TestBaseSingleBrowserROS.test_ros_position_after_search}\pysiglinewithargsret{\bfcode{test\_ros\_position\_after\_search}}{\emph{*args}, \emph{**kwargs}}{}
Run browser and type something in the search box, a place with
a known position.

Helper background process listening on \textbf{/portal\_kiosk/current\_pose}
ROS topic and checks ROS position messages until the one expected
arrives (within certain timeout).

Pass/fail flag is set accordingly by the listener process via
shared memory value which this test cases evaluates eventually.

\end{fulllineitems}


\end{fulllineitems}

\index{TestBaseTwoBrowsersROS (class in test\_ros)}

\begin{fulllineitems}
\phantomsection\label{test_ros:test_ros.TestBaseTwoBrowsersROS}\pysigline{\strong{class }\code{test\_ros.}\bfcode{TestBaseTwoBrowsersROS}}
Tests involving two browsers.

General idea is to selenium-manipulate one browser (with kiosk
extension) and assert accordingly adjusted state (as propagated
through ROS) in the other browser (display extension).
\index{test\_ros\_positions\_in\_browsers\_aligned\_after\_kiosk\_search() (test\_ros.TestBaseTwoBrowsersROS method)}

\begin{fulllineitems}
\phantomsection\label{test_ros:test_ros.TestBaseTwoBrowsersROS.test_ros_positions_in_browsers_aligned_after_kiosk_search}\pysiglinewithargsret{\bfcode{test\_ros\_positions\_in\_browsers\_aligned\_after\_kiosk\_search}}{}{}
Perform search in the kiosk browser and assert on the automatically
synchronized final position in the display browser.

\end{fulllineitems}


\end{fulllineitems}



\section{test\_runway module}
\label{test_runway:test-runway-module}\label{test_runway:module-test_runway}\label{test_runway::doc}\index{test\_runway (module)}
Tests related to Runway elements.

The Runway is the element at to bottom of the touchscreen browser,
it contains a list of Points of Interests
(if the camera is zoomed in to some level, and there are some
interesting places around) or the list consisting of the Earth, Moon, Mars.

There is also a list of Famous Places (Earth tours), which is always
filled (loaded by a static list of entries read from a file).
\index{TestRunway (class in test\_runway)}

\begin{fulllineitems}
\phantomsection\label{test_runway:test_runway.TestRunway}\pysigline{\strong{class }\code{test\_runway.}\bfcode{TestRunway}}
Runway element interaction tests.
\index{prepare\_poi() (test\_runway.TestRunway method)}

\begin{fulllineitems}
\phantomsection\label{test_runway:test_runway.TestRunway.prepare_poi}\pysiglinewithargsret{\bfcode{prepare\_poi}}{}{}
Prepare for Points of Interest tests.

Load browser, search for a location with POIs and wait
some time for the runway tray to populate.

\end{fulllineitems}

\index{test\_runway\_buttons\_basic() (test\_runway.TestRunway method)}

\begin{fulllineitems}
\phantomsection\label{test_runway:test_runway.TestRunway.test_runway_buttons_basic}\pysiglinewithargsret{\bfcode{test\_runway\_buttons\_basic}}{\emph{*args}, \emph{**kwargs}}{}
Test that Point of Interest and Famous Places are displayed.

\end{fulllineitems}

\index{test\_runway\_check\_earth\_icon\_click() (test\_runway.TestRunway method)}

\begin{fulllineitems}
\phantomsection\label{test_runway:test_runway.TestRunway.test_runway_check_earth_icon_click}\pysiglinewithargsret{\bfcode{test\_runway\_check\_earth\_icon\_click}}{\emph{*args}, \emph{**kwargs}}{}
The Earth icon (most left picture), clicking it should bring the
view to a considerably zoomed out position.

NB: position object values differ between subsequent runs.

\end{fulllineitems}

\index{test\_runway\_planets\_on\_max\_zoom\_out() (test\_runway.TestRunway method)}

\begin{fulllineitems}
\phantomsection\label{test_runway:test_runway.TestRunway.test_runway_planets_on_max_zoom_out}\pysiglinewithargsret{\bfcode{test\_runway\_planets\_on\_max\_zoom\_out}}{\emph{*args}, \emph{**kwargs}}{}
Test there is Mars, Earth, Moon loaded
in Points of Interest tray on maximal zoom out.

\end{fulllineitems}

\index{test\_runway\_points\_of\_interest() (test\_runway.TestRunway method)}

\begin{fulllineitems}
\phantomsection\label{test_runway:test_runway.TestRunway.test_runway_points_of_interest}\pysiglinewithargsret{\bfcode{test\_runway\_points\_of\_interest}}{\emph{*args}, \emph{**kwargs}}{}
Test Points of Interest is loaded, we can load one, exit it and
and load another Point of Interest.

\end{fulllineitems}


\end{fulllineitems}



\section{test\_zoom module}
\label{test_zoom::doc}\label{test_zoom:module-test_zoom}\label{test_zoom:test-zoom-module}\index{test\_zoom (module)}
Tests related to Zoom buttons, zoom operation.
\index{TestZoomButtons (class in test\_zoom)}

\begin{fulllineitems}
\phantomsection\label{test_zoom:test_zoom.TestZoomButtons}\pysigline{\strong{class }\code{test\_zoom.}\bfcode{TestZoomButtons}}
Tests for checking the zoom buttons are functional.
\index{test\_zoom\_buttons() (test\_zoom.TestZoomButtons method)}

\begin{fulllineitems}
\phantomsection\label{test_zoom:test_zoom.TestZoomButtons.test_zoom_buttons}\pysiglinewithargsret{\bfcode{test\_zoom\_buttons}}{\emph{*args}, \emph{**kwargs}}{}
Test that the zoom in and out buttons are displayed.

\end{fulllineitems}

\index{test\_zoom\_in\_button\_change() (test\_zoom.TestZoomButtons method)}

\begin{fulllineitems}
\phantomsection\label{test_zoom:test_zoom.TestZoomButtons.test_zoom_in_button_change}\pysiglinewithargsret{\bfcode{test\_zoom\_in\_button\_change}}{\emph{*args}, \emph{**kwargs}}{}
Test clicks on the zoom out button and checks the
pose object coordinates (altitude, latitude, longitude)
according change.

\end{fulllineitems}

\index{test\_zoom\_out\_button\_change() (test\_zoom.TestZoomButtons method)}

\begin{fulllineitems}
\phantomsection\label{test_zoom:test_zoom.TestZoomButtons.test_zoom_out_button_change}\pysiglinewithargsret{\bfcode{test\_zoom\_out\_button\_change}}{\emph{*args}, \emph{**kwargs}}{}
Test clicks on the zoom in button and checks the
pose object coordinates (altitude, latitude, longitude)
according change.

\end{fulllineitems}


\end{fulllineitems}



\renewcommand{\indexname}{Python Module Index}
\begin{theindex}
\def\bigletter#1{{\Large\sffamily#1}\nopagebreak\vspace{1mm}}
\bigletter{t}
\item {\texttt{test\_display\_kiosk}}, \pageref{test_display_kiosk:module-test_display_kiosk}
\item {\texttt{test\_general}}, \pageref{test_general:module-test_general}
\item {\texttt{test\_google\_menu}}, \pageref{test_google_menu:module-test_google_menu}
\item {\texttt{test\_platform}}, \pageref{test_platform:module-test_platform}
\item {\texttt{test\_ros}}, \pageref{test_ros:module-test_ros}
\item {\texttt{test\_runway}}, \pageref{test_runway:module-test_runway}
\item {\texttt{test\_zoom}}, \pageref{test_zoom:module-test_zoom}
\end{theindex}

\renewcommand{\indexname}{Index}
\printindex
\end{document}
